\section{Incipit}
Nella progettazione di una base di dati ci sono aspetti essenziali da prendere in considerazione per assicurare un servizio quanto più efficiente possibile.
Uno di questi servizi è certamente la \emph{qualità dei dati}, una base di dati con questa caratteristica farà si che le inconsistenze tra i dati siano il minor numero possibile.
Negli ultimi anni la crescita delle reti ha portato ad un aumento considerevole del flusso di dati  rendendo la \emph{data quality} una materia estremamente interessante vista la cospicua presenza di dati "sporchi" proveniente da fonti differenti.
Per ridurre questo tipo di anomalie è impensabile tentare di eliminare le \emph{inconsistenze} manualmente,  una procedura di questo tipo può essere facilmente incline ad errori sopratutto con la quantità di dati precedentemente citata.
In questo lavoro ci vengono incontro le \emph{Dipendenze funzionali}, utilizzate ampiamente per stabilire vincoli di integrità tra i dati e ridurre anomalie e inconsistenze all'interno della nostra base di dati.
La grande mole di dati, però, ha reso necessario un riadattamento delle dipendenze funzionali rendendole in grado di catturare inconsistenze più ampie nei dati. 
Le \emph{Dipendenze funzionali rilassate o approssimate} \textbf{(RFD)} sono da considerarsi come una naturale evoluzione o generalizzazione delle \emph{dipendenze funzionali canoniche}.
Questo nuovo strumento ci permette di adattare le semplici dipendenze funzionali a diversi contesti applicativi, infatti, le RFD possono applicarsi anche solo ad una porzione di database.
Il concetto più importante introdotto dalle RFD, però, è quello della \emph{similarità}.
Nelle dipendenze funzionali classiche esisteva soltanto il concetto di uguaglianza tra dati, nelle RFD espandiamo questo concetto ad una similarità, questo ci permetterà di coprire una quantità di dati maggiore e sfruttare le RFD appena scoperte per effettuare una operazione di \emph{cleaning} sulla base di dati.
Tuttavia le RFD possono fornire vantaggi solo se possono essere scoperte automaticamente.
Il lavoro di tesi si è basato su questo ultimo concetto di ottenere le RFD in seguito ad una procedura automatizzata.
Durante le varie fasi di studio si è pensato ed implementato un algoritmo che permette, attraverso tre fasi intermedie, la scoperta di RFD di un dataset dato come input.
Le tre fasi di questo algoritmo sono: \emph{Feasibility,Minimality,RFD Discovery} .
Per questo lavoro di tesi mostreremo l'idea dell'algoritmo generale ed entreremo nel dettaglio della seconda fase di sviluppo(Minimality), mostrando, infine, i risultati della sperimentazione.
Per questo algoritmo, particolare attenzione è stata posta sull'efficienza, oltre che sull'efficacia, studiando un'implementazione basata sul multithreading e predisponendola ad eventuale adattamento parallelo.
