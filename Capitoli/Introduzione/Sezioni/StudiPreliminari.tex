\section{Studi preliminari}
Prima di iniziare lo sviluppo dell'algoritmo per la scoperta di RFD si è reso necessario uno studio approfondito di un algoritmo precedentemente sviluppato per un progetto di IA \cite{tesinaIA}.
Tale algoritmo è stato sviluppato in Python, pertanto, abbiamo effettuato uno studio del linguaggio precedentemente citato.
Oltre le principali caratteristiche di questo linguaggio, è stato fatto uno studio anche delle librerie utilizzate all'interno del progetto:
\begin{itemize}
	\item \textbf{\emph{Pandas}}: È una libreria che include delle strutture dati e tool di analisi facili da usare e fortemente ottimizzate. 
	\item \textbf{\emph{Numpy}}: È un package dedicato all'elaborazione scientifica sul linguaggio Python.
\end{itemize}
Una volta concluso questo tipo di studio si è cominciato a pensare allo sviluppo dell'algoritmo in un ambiente differente.
La scelta è ricaduta su \emph{Java}, tale scelta è dovuta, oltre che alla già piena conoscenza del team di questo linguaggio, alla potenza e versatilità che questo linguaggio ci offre, oltre che al gran numero di framework presenti per la gestione di parallelizzazione e concorrenza, essendo quest'ultimo un aspetto molto importante per l'efficienza dell'algoritmo.
Le librerie esterne studiate ed utilizzate saranno ben approfondite nel capitolo 4(\emph{Implementazione}) di questo elaborato.
Le sopracitate librerie esterne utilizzate sono:
\begin{itemize}
	\item \textbf{\emph{AKKA}}: Framework per la  gestione del parallelismo e concorrenza.
	\item \textbf{\emph{Joinery Dataframe}}: Struttura dati simile al dataframe presente in \emph{Pandas} di Python.
\end{itemize}
All'infuori delle conoscenze legate ai linguaggi di programmazione, è stato necessario leggere e studiare vari documenti legati al mondo delle dipendenze funzionali. 