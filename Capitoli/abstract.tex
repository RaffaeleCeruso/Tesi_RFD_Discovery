Negli ultimi anni la crescita delle reti ha portato ad un aumento del flusso di dati rendendo così necessario uno sforzo maggiore per catalogare, indicizzare e pulire i dati. Quando si progetta una base di dati si tiene conto di alcuni importanti parametri che ne attestano la qualità, come le dipendenze
funzionali che sono utilizzate per ridurre le anomalie e migliorare la qualità dei dati. Le dipendenze funzionali sono utilizzate anche per altri scopi che si sono resi evidenti nell’ultima decade. Infatti, all’aumento della quantità di dati verificatosi negli ultimi anni è conseguito un aumento inversamente proporzionale della qualità. E’ stato pertanto necessario adattare le dipendenze funzionali per individuare le inconsistenze in modo più ampio. Le dipendenze funzionali rilassate o approssimate (RFD) sono una generalizzazione delle dipendenze funzionali canoniche che le rendono facilmente adattabili ai diversi contesti applicativi.Infatti una RFD pu`o applicarsi solo ad una porzione di un database e, cosa più importante, il concetto di uguaglianza tra valori di tuple è sostituito nelle RFD con il concetto di similarità. Tuttavia le RFD possono fornire vantaggi solo se possono essere scoperte automaticamente dai dati.