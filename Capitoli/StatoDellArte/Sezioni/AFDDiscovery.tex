Una \textit{dipendenza funzionale approssimata}(AFD) è una canonica FD che deve essere soddisfatta da \textquoteleft più' tuple, piuttosto che \textquoteleft tutte', di una relazione $r$. In altre parole, una AFD permette a una piccolissima porzione di tuple di $r$ di violarla. Diversi approcci sono stati proposti per calcolare il grado di soddisfacibilità di una AFD. Gli approcci principali sono basati su una piccola porzione di tuple $s \subset r$ per decidere se una AFD esiste su $r$. Come conseguenza, le AFDs che esistono su $s$ possono anche esistere su $r$, con una data probabilità. Alcuni metodi sfruttano la misurazione dell'errore della super chiave per determinare la soddisfacibilità approssima delle AFDs.   
